\documentclass{article}

% Packages
\usepackage{indentfirst}
\usepackage{hyperref}
\usepackage{hyperref} \hypersetup{
  colorlinks=true,
  linkcolor=black,
  citecolor=black,
  urlcolor=blue,
}

\title{%
  CSCI 5481 Final Project \\
  \large Quantum machine learning in bioinformatics
}
\author{Brian Cooper \\ coope824@umn.edu \\ University of Minnesota}
\date{\today}

\begin{document}
\maketitle

\section{Abstract}
  Quantum computing is an emergent field that provides promising potential for the field of bioinformatics. Among various problems, classification is a common task in bioinformatics machine learning tasks for identifying and predicting classes of objects, such as organisms or genes. Using a quantum variant of the classical support vector machine, a quantum model can be created in a similar manner to a classical support vector machine model for making predictions. Applied quantum computing is quite new, and it is not commonly used in modern bioinformatics experiments. Here, it is demonstrated that quantum computing frameworks, although in their infancy, can be used to design and test classification models, and are robust enough to prototype. The accuracy of the quantum support vector machine is demonstrated to compare competitively with its classical counterpart, despite claims that quantum computing is not yet mature enough for widespread use. This challenges the current stigma associated with quantum computing. These results demonstrate that quantum computing may be ready for testing in modern experiments extending beyond bioinformatics into other scientific domains.

\section{Related Work}
  Abc (explain previous findings here)

\section{Methods}
  Qiskit provides two distinct methods for performing quantum computing experiments:

  Multivariate data was chosen from the UCI Machine Learning Repository~\cite{}.

  \begin{itemize}
    \item{Real quantum computing}
    \item{A built-in simulator for real hardware}
  \end{itemize}

  I opted for the second option, as I do not have any academic credit required to perform experiments on real quantum computing hardware. \\

  A support vector machine with a radial-basis kernel was chosen

\section{Results}
  Abc

\section{Conclusion}
  Abc

\section{Supplementary Material}
All of the source code is publicly available at [URL].

https://archive.ics.uci.edu/ml/datasets/Yeast
https://archive.ics.uci.edu/ml/datasets/Ecoli
https://archive.ics.uci.edu/ml/datasets/gene+expression+cancer+RNA-Seq
https://archive.ics.uci.edu/ml/datasets/Mice+Protein+Expression
https://archive.ics.uci.edu/ml/datasets/PubChem+Bioassay+Data

\end{document}
